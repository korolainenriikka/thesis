\documentclass{article}
\usepackage{graphicx}
\usepackage{amsmath}
\usepackage{amsthm}
\usepackage{amssymb}
\usepackage{geometry}
%\usepackage[style=numeric,bibstyle=numeric,backend=biber,natbib=true,maxbibnames=99,giveninits=true,uniquename=init]{biblatex}
\usepackage[utf8]{inputenc}

\title{Fine-tuned optical character recognition for dental fossil markings}
\author{Riikka Korolainen}
\date{Sep 10 2024}

\begin{document}

\maketitle

\section{Abstract}
%Maximum of 2000 characters for the main text
%No figures or tables are allowed
%Maximum of 4 references in the text, which should be listed at the end


%(background) 1 sentence: background
% digitize and uniformize structure of handwritten fossil record will greatly increase sample sizes, leading to more accurate analysis of paleontological data.
%90, 000 + written fossil records in knm, need to be inserted to a database for better accessibility of data
Digitizing and uniformizing the structure of handwritten fossil records exhibits a great 
potential for increasing the accuracy of paleontological data analysis by increasing sample sizes. 
Approximately 90, 000 of such samples reside in the archives of the National Museum of Kenya, and 
an ongoing effort is to store this data in a digital format for better accessibility.
%(what has been done)
%digitization with a commercial ocr model
%generalist reader model does not know special tooth notation characters, which causes loss of information on eg upper or lower jaw, left or right
%and is not able to utilize prior knowledge on data containing fossil related vocabulary
A previous project utilized a commercial optical character recognition service for automated reading of these catalogues. This generalist
handwriting detection model lacked the ability to detect special characters used to denote tooth samples, and could not utilize prior knowledge 
of the vocabulary that is more likely to be present in the data, leading to loss of information and detection mistakes.
%what is done (scope)
%state of the art optical character recognition model is fine-tuned with few-shot transfer learning and 
%handlabeled fossil data to be better at interpreting the fossil catalogue
%compare character recognition accuracy of vairants of  cnn \& transformers (eg trocr) combined with previously successful transfer learning methods (eg tibetan script)

This thesis aims to build a specialist character recognition model to increase the accuracy of 
the bone or tooth type specifying column of the digitized data by fine-tuning a state-of-the-art optical 
character recognition model with few-shot transfer learning. This is performed by first finding most accurate
recognition models, variants of convolutional neural networks or vision transformers, and most successful 
transfer learning methods for adapting a model to a new character set. Then, the character 
recognition accuracy of combinations of these methods are benchmarked using handlabeled image segments from the 
fossil catalogues. The final aim of this work is to use the best-performing model 
to obtain an accurate reading of the catalogues of the National Museum of Kenya, and publish the final model to be used 
by the paleontological community for further digitization efforts.

%what having this done enables (conclusion)
%aim of this is to significantly improve the precision of element column, which body part / tooth part it is, in the digitized dataset

\section{Keywords}
%Include 4-6 keywords at the end of the abstract
Optical character recognition,
Few-shot transfer learning,
Vision transformers,
Convolutional neural networks,
Paleontological databases

\section{Address, acknowledgements}
%Please list all relevant coauthors and include institutional addresses
%Acknowledgements are allowed and encouraged at the end of the abstract
Institutional Address:

University of Helsinki, Faculty of Science
P. O. Box 68 (Pietari Kalmin katu 5)
00014 University of Helsinki

Acknowledgements: The previous project on digitization of 
the handwritten fossil catalogues was supervised jointly by Kari Lintulaakso from the natural history museum LUOMUS 
in Helsinki and collection curator Stephen Maikweki from the National Museum of Kenya.
The student group working on the project was Max Väistö, Riikka Korolainen, Janne Tuukkanen, Yinong Li and Axel Wester.
The initial phases of the thesis work have been instructed by professors Indrė Žliobaitė and Arto Klami from the Department 
of Computer Science at the University of Helsinki.
\end{document}
