\documentclass{article}
\usepackage{graphicx}
\usepackage{amsmath}
\usepackage{amsthm}
\usepackage{amssymb}
\usepackage{geometry}
\usepackage{subcaption}
\usepackage[style=numeric,bibstyle=numeric,backend=biber,natbib=true,maxbibnames=99,giveninits=true,uniquename=init]{biblatex}
\usepackage[utf8]{inputenc}

\addbibresource{../bibliography.bib}
\title{Thesis notes}

\begin{document}

\tableofcontents

\section{Abstract}

Keywords: Optical character recognition, Few-shot transfer learning, Vision transformers, Paleontological databases

\section{Introduction}
 
% use 'we' in intro

% outline of an introduction
% introduce the broad research area, why this is interesting. 1-2 paragraphs. context, anyone should be able to understand.
% first sentences: state the topic clearly
broad research area: paleontology: data analysis on fossil finds
dig fossil from the ground. identify:which bone,species,time. write this on a field slip, a slice of
baking sheet like paper 
analysis: take set of fossils, use methods for deducing eg climate, habitat, vegetation
why this is interesting 
    reactions of ecosystems to climate change
    what ancient worlds were like 
    how ancient humans lived
    mass extinction events

now we have a stack of baking sheets in a room in kenya.
paleontologists form all over the world want to solve climate change, among other problems
big data methods would explode what paleoecology can do
so we need to put the baking sheets on the computer to do analysis on big data
what is the topic of my work?
the baking sheets contain weird characters that a normal reader cannot read.
my topic is to read them

% explanation of the specific problem
given scanned images and data with bounding boxes of sentences and words where tooth denoting words are 
badly read, how can tooth element recognition results be improved? Goal is to have both this is what the element 
/ nature of specimen column says (eg example here) and what/which teeth are found in this specimen in standardized format
(eg example here)

% brief review of standard solutions to this (or related) problem(s) and their limitation in this case (incl. reference key papers)
% TODO, once literature review is done

% outline of the new solution
% TODO, likely: tooth or not classifier -> trocr or tooth classifier -> output concatenation

% how the solution was evaluated, what were the results of this evaluations. scope and limitations
% TODO
limitations: compute (no HPC utilized), small amount of labeled data, scope is masters thesis so very advanced techniques are not feasible.

% relevance for other work: why was this specific problem? how can this be concretely used?
relevance of this work: KNM is able to have way more precise dental element markings
to other catalogues: previous project + this a complete solution to digitizing the handwritten data 
relevance of this work: any field that does:
- ocr on unconventional characters
- ocr where each character has a multivariate output (eg. this is an a. it is underlined could be letter and underlined /not underlined)

The direct impact of this work is an improved precision of the tooth element entries in the digitized 
fossil catalogues of the National Museum of Kenya, but the results are applicable to a wider domain of problems.
Intuitively, the results are directly applicable to other fossil archives using similar notation: only a fine-tuning of the 
models to the new archive data is necessary.
For other handwritten archives, the results presented can be used to improve recognition accuracy, especially in cases 
where the data contains characters other than latin letters or arabic numerals. Additionally, this work presents
a potential solution for when the target character set can be expressed with multivariate output data. This could, for 
instance, be handwriting with occasional underlinings, where the bivariate output could be the letter and a boolean variable for 
whether the character was underlined.

% organization
The rest of this thesis is organized as follows. First, the necessary background theory is 
presented. For deep neural networks the following concepts are introduced:
the basic network structure, how training is conducted, basic building blocks of character-recognizing network architectures, performance-improving 
heuristics, and transfer learning. For paleoecology, the background covers 
foundational ecological laws followed by a brief introduction to methods used in
paleoenvironmental reconstruction, especially focusing on inferences from tooth data.
As the last background section, the composition of mammal teeth is presented. 
Second, related work is presented, both on
handwritten archive digitization and transfer learning with character-recognizer models.
Next, the experimental setup is introduced, covering dataset creation, labeling and data 
preprocessing, followed by base model and transfer learning method selection. After this, 
results of the experiments are presented and discussed. Finally, the work is concluded.

\section{Fundamentals on paleoecology}

% start of section metatext
% how this relates to the whole this is the application area
% so understanding the application area maybe gives insights to why things are done like this

Nature is highly complicated -> models, approximate models and assumptions enable drawing conclusions from
known distributions of species. 

each assumption / model can be questioned but they hold as a rule. only 
models briefly presented here, all statements here can be questioned to some extent 

idea: what fossil/dental data can be used for. how fossil/dental data is used

to highlight why accurate, fine-resolution (ie specific) and large magnitude of 
fossil, esp dental fossil data is genuinely useful.

chapter overview: review ecology and assumptions the analysis is based on. 
then, short overview of main techniques for paleoenvironmental reconstruction, the main 
application area of fossil data. last, mammal teeth row is presented to introduce terminology 
present in the data.

\subsection{Basics on ecology}

basic laws: theory that the data analysis relies on

Tolerances and niches (fundamental + realized): basis for environmental reconstruction \cite{Faith_Lyman_2019} ch 2

tolerance = range of an environmental variable that is hospitable for the species, 
eg. imaginary small mammal (come up with some imaginary name) can live when avg temperature is +10-+15.
Niche = set of tolerances the species has. fundamental niche = possible environments for the species, 
realized niche = where it lives. center of tolerances is better than the edge (ch2)

main assumption uniformitarism (the fact that tolerances constraint things has not changed)
niche conservatism: Assume that nearest living relative has same tolerances now -> get past environment (ch3, lyman 2017),

this presents a mapping from taxa to environmental variables -> basis of analysis 
of past environments.

modern alternative to this: transfer functions: mappings from taxa data to enviroment 
learned using machine learning / statistical models (ch9, birks 1995)

benefit instead of tolerances/niches: they have subjective interpretation problems (book ch 9)

esp. teeth: dental ecometrics = inference of transfer functions given dental data (ch9 liu et al 2012)

next, turn to how to solve the problem of information to environment given the 
taxomic information to environmetal indicators mapping

\subsection{Paleoenvironmental reconstruction}

why: get information of what is to come with climate change (faith ch2)

definition of the problem: whan ancient habitats were like and what changes they 
underwent at which times (ch2)

overview main tehcniques: presence/absence, abundance, taxon free, diversity based,
size clines

presence absence (ch5):
dataset is list of taxa that are present. absence is also indicator 
but worse since might be that the species just was not preserved / found. two approaches: 
fix location or fix set of species.
fix species: find where this set of species lives now (climate maps \& areas of sympatry), this indicates 
that historical locality had this climate. place-fixing: analyze how which species show up in this place
 changes over time, reduce species showing up data to lower dimension
(like pca), eg how many warm-climate vs cold-climate species show up, this gives ideas on changes in climate over time

abundance 
get relative NISP (number of species in sample): percent of species in sample is of this taxon (grayson 1984b, ch6)
do like presence absence but weigh signal according to abundance
and assume abundant species lives more toward center of tolerance 
grayson 1981: fossil accumulation affects datasets -> use with caution

taxon free 
cornelis van der klaauw 1948 (p 160) relation of traits of animals and livelihood = ecomorphology:
 eg what the animal eats also eg how diverse the place is. for environment mostly diet -> what plants grew and habitat -> climate etc
helps with not having to assume that closest relative tolerances were the same. (ch7)
eg usage of teeth: dental microwear. calandra and merceron 2016: analyze miniscratches on the surface to get diet to get vegetation and climate
    oma: to conduct this you need many samples of the same bone
    need to know exactly which bone it is since different teeth used for different things

diversity and size clines as enviromnental indicators
andrews et al 1979: how diverse also has a mapping to environment eg tropical is more diverse than arctic 
coarse indicators of the environment
lyman 2008: strong sample size effects: bigger sample is more diverse
so to do this you need equal size samples from all time periods analyzed
size clines basic idea: coarse indicators for size of bone -> environment eg larger libs warmer climate (mayr 1970)
also dental measures correlate with environment, eg mandibles (faith et al 2016). for that you need sufficient 
tooth samples and correct identification which tooth it is

end lesson: sample sizes and data precision are important. therefore it is superimportant 
to get more data / improve accuracy, which is the main goal of the digitization effort.
scale sense: faith lyman ch 4 said 1000 is solid 10 000 whopping sample size
dataset in question in this work has 90,000 so the value is pretty clear

\subsection{Composition of mammal teeth}

Fossils occur when animal / plant remains are deposited in a sediment in a way that preserves 
some part of its original form. Since teeth are the hardest material in animals, large fraction
of found parts are teeth. Fossil finding is followed by identification to most specific taxon possible
largely a technical skill (ch5), teeth are identified down to type and number, how manyeth the teeth are,
counting from center to edge or other way round??
specimen can be either one tooth or fragments of the jaw bone where there are multiple teeth (markings like M1-3)

from \cite{Hillson_2005} what teeth are composed of

 the jaw bones
lower jaw bones: mandibles
 permanent and deciduous (D), nonpermanent "milk" teeth (laita vaan jos löytyy d-hampaita)

right and left sides are always symmetrical, denoted simply L or R or Lt or Rt or left or right. left is left looking from the animal, not the observers perspective
Identicality also causes that sometimes tooth fossils are misidentified to the wrong side and corrected (ei lähteestä vaan nähty datasta koska l ja r on sutattu aika monta kertaa ja vaihettu

four classes, front to back: three incisors (I), one canine (C), four premolars (P), three molars (M). top bottom left right. top/bottom noting upper jaw as superscript lower jaw as lower script, 
 purpose: incisor -> catching, canine -> stabbing / killing prey, molars are for chewing. premolars are bit like canines bit like molars, function varies lot
 between taxa including holding, cutting and chewing. also form and number of each present changes between taxa.
sometimes lower jaw as line on top and upper jaw as line on bottom, sometimes both are used: upper script number with line on bottom. Line is "the other jaw"
if there are less of a type of teeth eg two premolars, they might be no 1 and 2 or no 3 and 4

% summary of chapter
% what to remember from this chapter
% it is important to have lots of precise data to do precise paleoecological analysis
% mammals have 3 molars 4 premolars 1 canine 3 incisors per side of jaw

\section{Deep Neural Networks for Optical Character Recognition}

% introduction to chapter
% point of this section: present relevant deep learning theory
% running example: reading characters from images (optical character recognition, OCR)

separate problems: character classification (easy, kNN, SVM), reading variable-length text (harder) \cite{ocr_survey}
introduce the problem of ocr, example: fossil catalogue

%miscellaneous keywords to explain (maybe somewhere) from \cite{li2021trocr}
%- knowledge distillation
%- model compression
%- image patches

\subsection{Deep Neural Networks}

- neurons and activation functions. maybe examples of activation functions: relu, sigmoid, softmax

what is a neural net
weights in layers: floating point numbers, grouped in groups 
activations: connections between weights, nonlinear scalar to scalar functions 

- feed forward
network computes output from input with the feed forward.
you have an input, bunch of numbers
then, you compute a linear combination and pass that through an activation function 

\begin{align}
h_d = a\left[ \theta_{d0} + \sum_{i=1}^{D_i}\theta_{di}h_{d-1} \right]
\end{align}

hd, hd-1 are hidden unit values. d is layer index. thetas are weights 
D is width of previous layer \cite{princebook}

a is any nonlinear funtion, simplest is the rectified linear unit relu, x=x if x>0, x=0 if x<0.
hidden units are taken as inputs to next layer and next and next. thetas = weights
largest, deepest: hundreds of layers, hundreds of millions of weights 
then last hidden unit output it the output of the network. tadaa.

- universal function approximator
the theory of the universal approximation capacity: this algebraic construct, 
given correct weights, activation functions and structure, could approximate 
any mapping from input to output. note: input/output dimension can be anything

examples from ocr relevant in this case
input is always image ie 2d matrix if grayscale or 3d tensor if rgb image.

classification problem, softmax activation
image of tooth sample (letter+number) to four classes, M P I or C.
input: image, output: four probabilities that sum to 1, probability this is a m
p i or c. sum to 1 is achieved with softmax. output is largest probability.

multilabel classification problem:
image of tooth sample. output: first MIPC, second if it is upper or lower jaw
output can be two arrays, one like before, other a 0-1 probability for upper jaw. upper if this > 0.5 

present sequence to sequence learning:
more complex case: image of sentence
output: text on image, variable length. here output layer is a more complex 
structure of probabilities for each position in the result sequence.

insert image example: inputs and outputs of different learnable functions with neural networks

NEW: I should probably introduce the softmax function here

\subsection{Training neural networks}

u have encoded the structure, starting weights. data, input/output pairs.
training = process that adjusts weights so that network becomes a good 
approximator

1. take part of data as training data
3 divide to batches, common batch sizes are exponents of 2, 2-32 usually 
4. pass a batch through the network. get output 
5. pass output and correct to loss function: map from output, correct to scalar,
0 is good, large value bad.
6. Compute loss function gradient with respect to network weights using automatic differentiation.
algorithm to get this in code is called backpropagation because it moves backward in the network
7. gradient informs how to adjust weights so that loss should decrease. 
adjust weights in this direction by preset amount called learning rate. Optimizers
are algorithms that determine the specifics of "moving in the direction of decreasing loss"
most common stochastic gradient descent (inserts randomness to movement) and 
adam (uses previous iteration movements known as momentum in process to make movements more smooth)
8. run  batches until out of data  = epoch. run many epochs, stop according to 
stopping condition that is known to be a state when the network weights are good.
goal: reach global minimum of loss function, difficult! would be perfect approximation
8. test on unseen test data to see generalization performance

lots of details on this process, rest is about that

\subsubsection{Loss functions}

more specific: how do you map output and label to scalar value describing how good the output was?
ocr point of view

Loss function is a function from model predictions and ground truth labels that describes with a single 
scalar value how good the match was, low number describing a good match \cite{princebook}.
These functions are constructed to be equivalent with maximum likelihood solution, think the model would 
output a conditional distribution of outputs, p(y|x).
each ground truth label in the training set should have a high probability in this distribution. Product of all 
these probabilities is called likelihood. Find parameters that maximize the likelihood of the training data set.
Loss functions are derived so that parameters bringing loss to zero is equivalent to the parameters with maximum likelihood.
Derivations are out of scope.

- cross-entropy loss 
kullback-leibler divergence of correct conditional probability and conditional probability parametrized by current model parameters.
(show formula, 5.27), correct is not dependent on parameters so is omitted. show 5.29, what is left from that 
(until here from \cite{princebook})

then: how cross-entropy loss is computed
% from https://machinelearningmastery.com/cross-entropy-for-machine-learning/, find a proper reference
used for classification problems. eg. is this letter in this image an 'a' or a 'b'.
correct probabilities eg .1 and .9 for a or b. model says .2 and .8. discretisized cross entropy computes 
it as .2*log.1+.8*log.9, log in base 2.

word detection models: have a predefined vocabulary, layer for probability of each word.
loss is cross entropy for these probabilities compared to target probability distribution, where correct word has probability 1 and 
all other have probability 0

- CTC loss

maybe

\subsubsection{Evaluating model performance}

common in character recognition: accuracy, f1 sometimes also used 
explain these and differences 

accuracy: correct out of those classified

why one uses precision and recall instead of accuracy
sometimes simple accuracy is not a good measure, eg cancer screenings: 1 in 500 has cancer.
then a dummy saying everyone is healthy would be correct 499/500 of the time, accuracy
99,8\%. measurements anyhow relevant would be 99.8-100, inconvenient range.

precision: fraction of positives identified. dummy would have precision 0
recall: out of those noted as positives, fraction of correct ones. dummy would have 0 too 
f1 is an average to lump these together in one number 

going forward: only consider simple accuracy scores as none of the classes are very rare 
in the fossil case.

\subsection{Architectures}

different ways of constructing layers, makes model pay more attention to desired things 
and reduces parameter count from fully connected layers, ie. encode priors \cite{alexnet}

relevant here: present the ImageNet competition that has initiated many new architectures.

\subsubsection{Convolutional layers}

% why convolve

encode a prior reduces the need to learn parameters. limit is cpu resources and data so given data 
and cpu, get as good model as you can requires constraining the problem by making assumptions \cite{alexnet}

prior encoded: move image a bit and it is still an image of the same object (translation invariance)
and nearby pixels are usually like each other (have a statistical relationship) fully connected nets 
do not consider input value positions in input vector, how near or far from each other they are, in any way.
\cite{princebook}

practical use: less parameters required -> less computational complexity

% what is a convolution, todo: notes
convolution (cross-correlation)
explain stride ie filter sparsity

image size on each layer is fixed \cite{vgg}

% size changes & pooling operations. todo: notes
max pooling / average pooling operations

% history bit: imageNet breakthrough models alexnet, googlelenet, vgg
historical scetch: alexnet \cite{alexnet} brought this to mainstream in 2012, imagenet 2014 saw GoogleLeNet \cite{googlelenet} and VGG \cite{vgg}.
these highlighted because the architectures are utilized in handwritten character classification.
alexnet, top5 error: 17\% best so far
whats new regularization with dropout, large model, large conv kernels (11x11-3x3 kernels), relu instead of previously popular tanhs sped up training, multiple GPUs used when training, normalization of convolution results, overlapping pooling (areas of pooling for each pixel overlap)
impact: brought deep learning to center of ml research, deep nets seem to outperform humans in feature engineering.

googlelenet \cite{googlelenet} won 2014 with top 5 error rate  6,75\%.
method: fixed number of allowed multiple-adds in forward pass, 
used inception (named after we need to go deeper meme) layer: 1x1 3x3 and 5x5 sparse convolutional kernels, also 
sparse fully connected layers ie not all weights are connected 
to save compute. Also inspired by biological visual systems. allowed
for deeper and wider network, turned out to be great in imagenet and 
object detection.

vgg \cite{vgg}. achieved 6,8\% with imagenet 2014 postsubmission, submitted 7,3\% thus 
second in competition. method: simple 3x3 convolutions, tested different depths. best depth 19.
new that previous best models did big kernels and shallower nets. also good model because of 
the simplicity + the paper included in appendix that demonstrated vgg as a feature extractor to 
be used in transfer learning.

\subsubsection{Transformers}

% !!! only do if you use transformers as a base model in the next section!

introduced by \cite{attention_is_all_you_need}

- self-attention
- multi-head self-attention
- tokenizing and the cls token

historical scetch: the vision transformer \cite{vit} outperformed many cnns in imagenet score when introduced and changed 
course of image classification research toward transformers

\subsubsection{Autoencoders}

- encoder/decoder

\subsection{Techniques and heuristics for improving performance}

data augmentation

dropout regularization (alexnet uses so need to intruduce)

\subsection{Transfer learning}

relation to the coarse level training loop: how to initialize model parameters

this sect from \cite{transferlearning_survey}. they initially formalized the problem and uniformized 
the terminology

basics: what it is, basic premise: if you start out from a parameter configuration that solves a related 
task well, there is lesser need to train to solve the new problem.

why: save compute (imagenet models vgg lenet alexnet training times 5 days to 3 weeks even when using GPUs \cite{vgg}) and data labeling (usual constraints in ml model building \cite{engbook})
also why: training many parameters on a dataset overfits model to the dataset, so freezing layers prevents overfitting \cite{googlelenet}

task (inductive) transfer: tasks differ and domain (transductive) transfer (only data differs)

task transfer relatedness + the more related the better results, negative transfer ie things made worse if 
things differ too much.

initialize weights to those that suit a related task, it is assumed that the starting point is already very good
catastrophic forgetting = forgetting the previously learned after finetuning

broad methods for inductive transfer: feature representation transfer, parameter transfer; priors or hyperparameters of model are assumed to be shared,
instance transfer: reuse source data (not considered), relatioal-knowledge-transfer: when data is not i.i.d (not considered here)
ont considered because does not apply here

feature representation transfer
seeing model as model doing the feature engineering: all but last layer are like 
feature extraction that allows a linear model to be fit from features to output. 
assumption for last layer transfer: reusing the feature extractor should bring good results with little computational cost

basically nowadays it is usually insensible to ever train from scratch (Stanford course notes, https://cs231n.github.io/transfer-learning/)
therefore this work definitely transfer learns. also previous work that does not transfer learn is dismissed, 
as it does not inform this case where target data is scarce

\subsubsection{Foundation models}

- generalist models
	- large unsupervised training data sets

Idea: since transfer learning source task performance is not important, target task is the main point \cite{transferlearning_survey}, 
you can come up with some nonsensical task such as map image to the same image to get unsupervised task, train 
massively on massive compute, and use the model for varieties of tasks. idea: do as little as possible in the target domain
Also the idea of unsupervised pretraining, is sometimes done on OCR.

% end of section
% this section presented relevant background on neural networks
% what you should remember from this chapter
% you have a network that can approximate any input to output mapping
% the network structure is set up in a way that is known to result in a good 
% approximation given the input to output mapping we try to achieve
% training is conducted by constructing output from input with the network
% computing difference of output and correct output with the loss function,
% and adjusting the network to a direction that reduces this difference.
% transfer learning means that you start this process from a network state
% that can map some related mapping well, so the optimal network for the new 
% state is closer.
% how to do this in the image to dental element describing phrase case is our topic

\section{Related work}

% intro
% this chapter presents related work from two angles
% angle 1: neural nets & transfer learning on similar problems
% angle 2: solutions to digitization of historical documents

Search strategy: few seed papers and snowball search. Related conferences: 
Frontiers in Handwriting Recognition

Notes on choices: there is math OCR and music OCR, but they use large datasets and no transfer learning, they dont suffer from the limited target data problem. Also, the problem domain is too different from my problem to be informative.

\subsection{Approaches to digitization of handwritten archival data}

obvious solution: sit down and type

other solutions: none, some digitization stuff exists but that is 3d imaging of physical specimens
or classifying images of specimens to eg taxa

\subsection{Approaches to character recognition with small target domain datasets}

method: snowball search. for a paper to be accepted for use here, i set up these requirements:
lists best accuracy clearly (percent, which dataset, number of classes)
specifies the base model used (architecture and source domain dataset)
specifies how transfer learning was conducted (not just used transfer learning)
then reputable venue and acceptable quality of text and presentation, super subpar text was a frequent reason to skip a paper, they also often lacked relevant details

\section{Experimental setup}

% introduction to chapter
% this chapter presents how the tests ran for this thesis were conducted

\subsection{Data description}

has been done by different annotators, no logs on who logged what, everyone 
had a bit different style of notating. also no clearly defined standard 
for notating specimens. so might be that actual data used will have 
characters or words not present in any data, causing errors.


Identicality also causes that sometimes tooth fossils are misidentified to 
the wrong side, seen in data with smudged over l's and r's

Catalogues have lines between entries. challenge for the model to not mark these as underlined
in these cases the line is long and spans the entire image, so it should be distinguishable from 
a single underlined character.

difference to other domains of big data:
each fossil sample is expensive and human laborous to obtain, thus "big" sample is 
small compared to other ml applications, 1000 is a lot 10k huge.

\subsubsection{Notes on creating the dataset}

Hand-labeled

Data was extracted from scans by getting bounding boxes from Azure Vision API,
finding the correct column (nature of specimen or element), and cropping the image 
according to bounding boxes.

Non-tooth samples were not discarded since they contain 
bone fossil related words and good samples of the handwriting style of this dataset.

smudged-over "L" was labeled as "R", and other way around: it seems that later 
someone found it was the opposite side after all. Hope of this is that the model 
would learn to map "messy L" as "R". snudged "left" or "right" was not noted as the 
opposite as there were too few such samples.

Superscript seems much more rare than lower script

Data was labeled not by individual characters but as full tooth descriptions
to preserve context where tooth special characters are more likely to occur

Some have been corrected by writing on top and thus are very hard 
even for humans to read, this is also an example of smudged-over correction: 

\includegraphics*[scale=0.2]{../images/superambiguous_data_sample.png}

\subsubsection{Unicode characters used for data labeling}

explain: unicode has graphenes with code points. eg a is one graphene one code point,
à is one graphene two code points (dot on top and the letter). the top thing -like characters will be called 
"modifiers".

markings contain letters and numbers with no line, line on top or line at the bottom.
Each character can be lower- or upper script. The modifiers used are: 
macron with lower ($\bar{\mathrm{A}}$) and upper variant.

Unicode \cite{unicode_homepage} has characters that are for example upper script, but 
these were not used for two reasons:

- lower and upper script character set is incomplete for this purpose (eg 3 with upper macron and lower script needed)

- from the model perspective 3 and $_3$ are no more similar than A and B, however, 
three combined with lower script modifier and 3 with upper script modifier 
all contain the same unicode character 3 with only the modifier changing. The 
problem here is that there is no lower or upper case modifiers in unicode. Therefore,
the caron ($\check{\mathrm{A}}$) was chosen as the lower script modifier, and the circumflex accent ($\hat{\mathrm{A}}$)
as upper script. These were chosen since the arrow-like modifier pointing up or down
is maybe the most logical placeholder for the missing modifier. More traditional 
workarounds of missing upper or lower script, the underscore "\_" and separate 
caret character "\^ " were not used to keep one unicode graphene represent one character 
on the page. Also on the other hand using one modifier for all lowercase characters allows 
the model to understand that there is a similarity between all lowercase characters.
The intention is that one idea about a character is encoded as one code point, so that 
the model can learn the mapping from the image of the character to the code point 
combination
(until now already in thesis text)
----

also: some annotators used / instead of line. left to / -> upper, right -> lower.
unknown/unsure noted with x with macrons on top/bottom
annotate with up/down macron

if model is toothornot classifier + tooth reader -> remove fractions notes

\subsection{Data preprocessing}

grayscaling, flipping to back background white foreground

I should especially do random cropping since that occurs in real dataset!

I sould do some border contrast decrease center contrast increase because border noise is present and usually less relevant 

all these: test first without, then add preprocessing step, then re-test

i am able to get labeled MPIC data by getting azure outputs where the letter is mpi or c

\subsection{Methods: base models and transfer learning tehniques}

\subsubsection{Problem formulation}

different types of defining inputs and outputs, discussion on what to choose

the sequence or character per character recognition question:
sequence is a mapping from image to variable-length phrase
character per character approach, inspired by \cite{tibetan_ocr}:
train a classifier: is this word a tooth or not? then give non-tooth to the untuned
trocr, which works very well. Then few shot transfer learn a classifier from an image
with only a tooth marking (letter and one number) to tooth. Possibly extend classifier to 
be able to recognize multiple teeth (eg M1-3). Target could be multivariate: first would
be tooth (which i1-3,c,p1-4,m1-3), second would be jaw (upper, lower, unknown), third side 
(left, right, unknown). Separating 'l/r/lt' from the letter+number tooth notation is 
fairly trivial: noncursive handwriting can be separated by finding a vertical line where there 
is no black. Image processing: convert to black and white, then find x coordinate with no black 
and split there.

sequence benefits: adaptable to many kinds and lengths of input, possible to get good inferences 
for surprising marking styles 
sequence bad sides: finetuning just one layer on 80 training images for two epochs took about 15 minutes
-> all hyperparameter optimization etc is out of question with this heavy training.
Also: not domain adaptation (adapting same task to different dataset), but task adaptation ie. 
the target set of characters has changed. Encoding this to the large encoder decoder transformers would 
require rewriting parts of the preprocessor and model which is too complex given the level of this work.

character by character benefits: feasible given available data and computing resources
possible to encode the classes (ie, teeth)
classifying characters has been essentially solved, easy problem 
also classifying to tooth or not tooth should be easy 
--> focus on model ensemble with tooth or not classifier + trocr + tooth classifier

multilabel class or classifier chains:
multilabel is difficult because the basic case has just set of labels and you can 
have any number of any of them. we have 4 types of classes with dependencies (no 4th canine)
so chains make sense, as constrained multilabel classification with class groups is a 
complex, nonstandard construct \cite{multilabel_classification}. so, mini model approach 

1. tooth or not: azure output text matches regexp letter and number or 'c' or 'C' -> 1/0 train using MPIC images
2. MPI classifier: tooth type (regexp already says c is c), upper lower classifier: 1/0
3. get number (1-4) from azure output, is generally correct by visual inspection

not inspired from any paper but my own idea, felt like a sensible way to formulate the problem, 
so we have for all models a well-solved well-defined basic classification problem.

reason for this choice: it encodes most prior knowledge in the output structure,
use as much from azure as possible

priors:
there is also the azure output for the tooth. it is likely to be very good but not constrained to the MPIC case.
include the azure label as one input variable
why we went for the simplest problem out there
because the part of data that can be digitized with the tightest constraints and the simplest models are most accurate. Do that first, deal with the remaining later
\subsubsection{Base model selection}

initial sequence learning attempts done with \cite{li2021trocr}, proof of concept so no real exhaustive comparison literature review

Chosen public dataset was EMNIST \cite{emnist} because it is the closest to my problem: a large dataset of labeled letters and numbers.

According to \cite{emnistclassifiersurvey}, this model was the best: \cite{jamilemnist}, was chosen as the base model.
they trained on emnist-balanced, so no difference between upper -and lower case letters. not a problem for us since 
in the handwritten catalogues, upper or lowercase was not used to distinguish upper and lower jaw

mnist and real world fundamental difference: mnist is perfect world, real images vary much more \cite{alexnet}
therefore: not e(mnist) as source problem, but imagenet classification.

\subsubsection{Transfer learning method selection}
Priors. base model already knows the output should be a word, eg "jdaslkjflkds" is a highly unlikely
correct answer.
Bone notation has a very small subset of possible english words, eg. the word "beach ball" cannot ever be a correct 
answer for a reading
--> these do not apply if we do character classification

% summary of chapter
% list all methods used for each aspect (preprocessing, base model, 
% problem formalization, transfer technique) chosen

\section{Results and discussion}

% introduction to chapter
% present the results from experiments

% summary of chapter
% takeaway points

table of results could contain:
rows: base models tested 
columns: transfer learning method 

transfer learning methods could include 

just training further (ie baseline, dont do anything special)

train the last layer (inspo from \cite{tibetan_ocr}, test hypothesis of reusing feature extractor)
what else? find from literature!

train unsupervised autoencoder on the target data (reduces domain transfer distance),
transfer to supervised classification task

cleaning data from original format with gaussian blur and otsu thresholding surprisingly did not 
change results much 

augmentation to balance classes resulted in a 10\% increase in tooth type classification accuracy 
using a simple translation. translation also makes sense since that is the main variable changing 
between images, zoom or rotation changing is more rare as these are scans.

initial phase experiments: mnist transfer to mpi since it was thought to be a similar problem. 
however like noted by \cite{alexnet}, real world image classification is much more complex than digit classification 
as the benchmark images are ideal cases with eg perfectly even background. Therefore later phases 
experimented with transfer from imagenet classifiers since that is what most of papers in related work also did.

\section{Conclusions}

% why word bounding box worked badly and how it could be improved
how this could be continued
main problem in working with azure bounding boxes: incorrect identification of word bounding boxes.
encoding prior knowledge should work better: M1-3 is a word for instance, azure did stuff like m, 1-, 3
from computer vision lecture 11 \cite{ruotsalainen2024}
object bounding box detection requires higher resolution images and ground truth labels -> labeling effort is 
big, compute needed. one idea is to take premade bounding boxes by generalist ocr word detectors and for example
infer they are correct by matching with a vocabulary of known correct words to exist in fossil catalogues.

% pointers to research work on computer vision object detection
I quickly tested a state of the art object detection model the yolo10, variant of popular object detector yolo\cite{redmonYouOnlyLook2016}
 untuned, on huggingface api, the yolo10 \cite{OmouredYOLOv10DocumentLayoutAnalysisHugging2023}. The performance 
was poor, see fig, probably since object detection usually does 3d object detection in a 3d scene, so one should either 
fine tune or find a ocr bounding box detector. then there is open-vocabulary object detector which you can tell what objects 
are present \cite{YOLOWorldRealTimeOpenVocabulary}. we could just say there are teeth and words. 
then also a very recent publication on detecting text on images \cite{longHierarchicalTextSpotter2024}.
benefit of these is that they are fine-tunable which the commercial azure api is not. 
then also detectron \cite{Detectron} from meta for object detection
one could also run more sophisticated 
tooth or not classification on azure output, which could work very well but relying on a paid service is not ideal 
for open access solutions.

% improving MPI/upperlower classification (not as hard so not as relevant perhaps)
ways to improve the result of image classification obtained here 
test svm after feature extraction instead of the dense layers + softmax. My most recent model was from (insert year),
since then imagenet classification has advanced, so try using a newer base model. these can be harder to obtain 
since they are not as established, I could just fetch models from torchvision.models, newer ones are not present there.
with a good object detector probably we can get to more variaty of tooth marking images (maybe a fig of examples?)
so the classification task would become more complicated. Still, the downstream tooth marking image to tooth task 
is a relatively easy classification problem comparing to eg imagenet, so the main challenge is definitely finding words and
classifying them to tooth or not.

% this requires thorough work
this was a very quick literature search and practical test, so proper thorough work should be done to verify if this speculation 
is correct or not.

\printbibliography

\end{document}