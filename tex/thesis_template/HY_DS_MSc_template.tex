%% This file is modified by Veli M�kinen from HY_fysiikka_LuKtemplate.tex authored by Roope Halonen ja Tomi Vainio.
%% Some text is also inherited from engl_malli.tex by Kutvonen, Erki�, M�kel�, Verkamo, Kurhila, and Nyk�nen.


% STEP 1: Choose oneside or twoside
\documentclass[english,twoside,openright]{HYgraduMLDS}
%finnish,swedish

\usepackage[utf8]{inputenc} % For UTF8 support. Use UTF8 when saving your file.
\usepackage{lmodern} % Font package
\usepackage{textcomp} % Package for special symbols
\usepackage[pdftex]{color, graphicx} % For pdf output and jpg/png graphics
\usepackage[pdftex, plainpages=false]{hyperref} % For hyperlinks and pdf metadata
\usepackage{fancyhdr} % For nicer page headers
\usepackage{tikz} % For making vector graphics (hard to learn but powerful)
%\usepackage{wrapfig} % For nice text-wrapping figures (use at own discretion)
\usepackage{amsmath, amssymb} % For better math
%\usepackage[square]{natbib} % For bibliography
\usepackage[footnotesize,bf]{caption} % For more control over figure captions
\usepackage{blindtext}
\usepackage{titlesec}
\usepackage[titletoc]{appendix}

\onehalfspacing %line spacing
%\singlespacing
%\doublespacing

%\fussy 
\sloppy % sloppy and fussy commands can be used to avoid overlong text lines

% STEP 2:
% Set up all the information for the title page and the abstract form.
% Replace parameters with your information.
\title{Template for Master's thesis}
\author{Riikka Korolainen}
\date{\today}
\prof{Professor X or Dr. Y}
\censors{Professor A}{Dr. B}{}
\keywords{Optical character recognition, Few-shot transfer learning, Paleontological databases}
\depositeplace{}
\additionalinformation{}


\classification{\protect{\ \\
\  Computing methodologies  $\rightarrow$  Machine learning $\rightarrow$ Machine learning approaches $\rightarrow$ Neural Networks \  \\
\  Computing methodologies $\rightarrow$ Machine learning $\rightarrow$ Learning paradigms $\rightarrow$ Multi-task learning $\rightarrow$ Transfer learning \  \\
\  Applied computing $\rightarrow$ Physical sciences and engineering $\rightarrow$ Earth and atmospheric sciences \  \\
}}

% if you want to quote someone special. You can comment this line and there will be nothing on the document.
%\quoting{Bachelor's degrees make pretty good placemats if you get them laminated.}{Jeph Jacques} 


% OPTIONAL STEP: Set up properties and metadata for the pdf file that pdfLaTeX makes.
% But you don't really need to do this unless you want to.
\hypersetup{
    bookmarks=true,         % show bookmarks bar first?
    unicode=true,           % to show non-Latin characters in Acrobat’s bookmarks
    pdftoolbar=true,        % show Acrobat’s toolbar?
    pdfmenubar=true,        % show Acrobat’s menu?
    pdffitwindow=false,     % window fit to page when opened
    pdfstartview={FitH},    % fits the width of the page to the window
    pdftitle={},            % title
    pdfauthor={},           % author
    pdfsubject={},          % subject of the document
    pdfcreator={},          % creator of the document
    pdfproducer={pdfLaTeX}, % producer of the document
    pdfkeywords={something} {something else}, % list of keywords for
    pdfnewwindow=true,      % links in new window
    colorlinks=true,        % false: boxed links; true: colored links
    linkcolor=black,        % color of internal links
    citecolor=black,        % color of links to bibliography
    filecolor=magenta,      % color of file links
    urlcolor=cyan           % color of external links
}

\begin{document}

% Generate title page.
\maketitle

% STEP 3:
% Write your abstract (of course you really do this last).
% You can make several abstract pages (if you want it in different languages),
% but you should also then redefine some of the above parameters in the proper
% language as well, in between the abstract definitions.

\begin{abstract}
Summary of the main contents of the work: topic, methodology and results.

Topics are classified according to the ACM Computing Classification System
(CCS): check command \verb+\classification{}+. A small set of paths (1-3) should be used, starting from any top nodes
referred to bu the root term CCS leading to the leaf nodes. The elements
in the path are separated by right arrow, and emphasis of each element individually can be indicated
by the use of bold face for high importance or italics for intermediate
level. The combination of individual boldface terms may give the reader
additional insight. 
\end{abstract}

% Place ToC
\mytableofcontents

\mynomenclature

% -----------------------------------------------------------------------------------
% STEP 4: Write the thesis.
% Your actual text starts here. You shouldn't mess with the code above the line except
% to change the parameters. Removing the abstract and ToC commands will mess up stuff.
\chapter{Introduction}

The thesis should have an introduction chapter. Other chapters can be named according to the topic. In the end, some summary chapter is needed; see Chapter~\ref{chapter:conclusions} for an example.

\chapter{Figures and Tables}

\section{Figures on teeth}

\section{Background}

\subsection{Neural Networks and Deep Learning}

\subsection{Fundamentals on paleoecology}

\subsubsection{Basics on ecology}

\subsubsection{Paleoenvironmental reconstruction}

\subsubsection{Diets and evolution}

\subsubsection{Composition of mammal teeth}

%Fossils occur when animal / plant remains are deposited in a sediment in a way that preserves 
%some part of its original form. Since teeth are the hardest material in animals, large fraction
%of found parts are teeth. Fossil finding is followed by identification to most specific taxon possible
%largely a technical skill (ch5), teeth are identified down to type and number, how manyeth the teeth are,
%counting from center to edge or other way round??
%specimen can be either one tooth or fragments of the jaw bone where there are multiple teeth (markings like M1-3)
% present teeth here

Since geological events tend to erode organic remains the faster the remain decomposes, the hardest materials in 
the corpse represent largest fractions of fossil datasets. These hard materials include shells, bones and especially teeth, and 
the last is prominent in fossil data analysis also due to the fact that they encode a diverse set of information on 
the livelihood of the organism \cite{Faith_Lyman_2019}. The identification of the fossil remain is done at the finest resolution possible,
preferring taxon information over just identifying the genus, for instance. Finest-resolution information 
derived from dental fossils are the taxon the tooth is from, and which tooth or teeth are found in the specimen.
This section presents the naming and indexing system for mammal teeth commonly used in paleontological datasets,
as described by Hillson \cite{Hillson_2005}, and some common shorthand versions present in the dataset digitized in this work.

% complete jaw-describing terminology
% the jaw bones
%lower jaw bones: mandibles, upper jaw: maxilla, premaxilla
% permanent and deciduous (D), nonpermanent "milk" teeth (laita vaan jos löytyy d-hampaita)
Specimens including more complete fragments of the jaw are described with terminology related 
to the jaw bones. All mammals share the same bone structure around the mouth: the lower jaw consists 
of two bones called \textit{mandibles}, joining in the middle, whereas the upper jaw consists of bones called 
\textit{maxilla} and \textit{premaxilla}, that also form large parts of the face.
A common trait across many mammals is also that the permanent teeth erupt in the 
youth of the animal, replacing the 'milk' or \textit{decidous} teeth. Shorthands commonly used for these 
terms are 'mand' for mandibles, and capital letter 'D' for the decidous teeth.

% types of mammal teeth
%four classes, front to back: three incisors (I), one canine (C), four premolars (P), three molars (M). top bottom left right. top/bottom noting upper jaw as superscript lower jaw as lower script, 
% purpose: incisor -> catching, canine -> stabbing / killing prey, molars are for chewing. premolars are bit like canines bit like molars, function varies lot
% between taxa including holding, cutting and chewing. also form and number of each present changes between taxa.
%sometimes lower jaw as line on top and upper jaw as line on bottom, sometimes both are used: upper script number with line on bottom. Line is "the other jaw"
%if there are less of a type of teeth eg two premolars, they might be no 1 and 2 or no 3 and 4
The tooth rows of mammals are classified to four classes; \textit{incisor}, \textit{canine}, \textit{premolar}
and \textit{molar} and indexed with a numbering system. Moving from the middle of the tooth row
towards the side, there are up to three 
incisors, used for catching food and denoted with the letter 'i'. Behind them is the canine tooth, used for cutting, and 
in case of predators, killing. This tooth is denoted with the letter 'c'. Behind the canine are up to four premolars, noted with 'p'. These 
teeth vary most between taxa in form and function with functions including cutting, holding and chewing food.
The teeth at the back of the row are called molars, 'm', and are primarily used for chewing. Molars, like the other tooth types, 
vary in number between taxa, and are at most three. The numbers are always increasing when moving back in the tooth row, but in
 the case of missing teeth in a taxon, the numbers do not necessarily start from one: instead, the number is chosen to 
have teeth with same numbers as alike each other as possible. Thus, a taxon with only two premolars might only have the teeth P3 and P4.


% directional terminology
% distal "far from center of body", proximal "close to center of body", mesial "close to mouth opening"
%right and left sides are always symmetrical, denoted simply L or R or Lt or Rt or left or right. left is left looking from the animal, not the observers perspective
Location of the tooth present in the fossil is described with directional terms specifying the side, jaw and the location on the jaw.
The most
intuitive are left and right describing the side, where one needs to note that each denotes the side from the viewpoint of the 
animal, not the observer. Mammal teeth are always symmetrical, thus every tooth always has the 
equivalent other-jaw counterpart. The distance of a tooth from the throat 
is described with the terms \textit{distal}, 'far from to the mouth' and \textit{mesial}, 'close to the mouth'. For skeletal bones, the term \textit{proximal}, 
'close to the center of the body' is often used instead of 'mesial'.
Short-form versions for these terms include capital 'L' or 'Lt' for left, capital 'R' or 'Rt' for right, 'dist.' 
for distal and 'prox' for proximal.
The jaw, upper or lower, has three dominant notation styles: one is to sub- or superscript tooth index numbers, other is to 
over- or underline tooth markings, and the last style, prominent in digital fossil data, is to set the tooth type letter to upper- or lowercase.
In each of these systems, a superscipt, underline, or capital letter denotes upper jaw, and conversely subscript, overline or lowercase letter denotes the lower jaw.
An illustration of the mammal tooth system is presented in Figure~\ref{image:mammal_teeth}. Terminology with corresponding shorthands are summarized in Table~\ref{table:terminology} and jaw notation styles in Table~\ref{table:jaw_notation}.

\begin{figure}[h]
    \centering
    \includegraphics*[scale=0.43]{../images/teeth_img_hillson_book.png}
    \caption{Mammal teeth composition, from Hillson \cite{Hillson_2005}.}
    \label{image:mammal_teeth}
\end{figure}

\begin{table}[ht]
    \centering
    \begin{tabular}{|l|l|l|}
        \hline
        \textbf{Term}       & \textbf{Meaning}                                   & \textbf{Shorthands}       \\ \hline
        Mandible            & Lower jaw bone                                     & mand.                     \\ %\hline
        Maxilla, Premaxilla & Upper jaw bones                                    &                           \\ %\hline
        Deciduous           & 'Milk teeth'                                       & D, d                      \\ %\hline
        Incisor             & Tooth type (front, middle)                         & I, i                      \\ 
        Canine              & Tooth type (between incisor and premolar)          & C, c                      \\ %\hline
        Premolar            & Tooth type (between canine and molar)              & P, p                      \\ %\hline
        Molar               & Tooth type (back of tooth row)                     & M, m                      \\ %\hline
        Distal              & Far from body center / mouth                       & dist.                     \\ %\hline
        Mesial              & Close to the mouth                                 &                           \\ %\hline
        Proximal            & Close to body center                               & prox.                     \\ \hline
    \end{tabular}
    \caption{Terminology related to mammal teeth with corresponding shorthands}
    \label{table:terminology}
\end{table}

\begin{table}[ht]
    \centering
    \begin{tabular}{|l|l|l|l|}
        \hline
        \textbf{Jaw}      & \textbf{Line Notation} & \textbf{Sub/Superscript Notation} & \textbf{Digital Notation} \\ \hline
        Upper         & $\text{M}^{\underline{1}}$      & m\textsuperscript{1}              & M1                        \\ 
        Lower         & $\text{M}_{\overline{1}}$ & m\textsubscript{1}                & m1                        \\ \hline
    \end{tabular}
    \caption{Dental marking styles, Example: first molar. Line notation displayed in common style combining sub- and superscripts.}
    \label{table:jaw_notation}
\end{table}

\section{Unicode notation}

The unicode system \cite{unicode_homepage} constructs all known characters as signs called graphenes.
Each graphene can consist of any number of code points, with each code point having an unique identifier, denoted with "U+code point id".
Examples of graphenes with one code point are latin letters, such as 'K', special characters, such as '@', '\%' and '+',
or letters from different writing systems, such as '$\omega$', '$\aleph$' or '$\mathfrak{A}$'.
 Examples of multi-code point graphenes 
are latin letters with accents, such as '$\hat{\text{e}}$', or emoji characters with non-default skin tone, such as {\twemoji{thumbs up: dark skin tone}}.
Code points added to the main code point, such as the circumflex accent '\^ ' are called modifiers.

The guiding principle in labeling the data was to encode each concept in the text as one unicode code point. A concept could be, for 
instance, the number two, or a character being positioned in subscript. The aim of this decision is to allow the model 
to find common image traits between characters of a similar type: a subscript character has dark pixels in lower positions, and shapes of all 
number two's have similar curvatures, for instance. As a second principle, it was chosen that each single character in the image, such as "letter C" 
or "a subscript four with a horizontal top line", would always be labeled as one graphene. 
These rules makes the encoding choices nonobvious: for example, 
a subscript number two would intuitively be labeled as the unicode code point '$_2$', but this was not done, 
since this graphene does not contain the code point for number two, 
and as a one code point graphene has no code point to extract to be used among the other subscript numbers.
Another intuitive choice, '\_2', would violate the one graphene per character rule.

\begin{table}[h!]
    \centering
    \begin{tabular}{|c|c|}
        \hline
        \textbf{Input Image} & \textbf{Label} \\
        \hline
        \includegraphics[width=0.3\textwidth]{../images/data_samples/canine.png} & Lt. \underline{C} frag. \\
        \hline
        \includegraphics[width=0.3\textwidth]{../images/data_samples/lowjawincisor.png} & Rt. I$\bar{\check{2}}$ frag. \\
        \hline
        \includegraphics[width=0.3\textwidth]{../images/data_samples/multipleteeth.png} & Lt. Md. frag. wt dP$\check{\bar{3}}$, P$\check{\bar{4}}$-M$\check{\bar{1}}$\\
        \hline
        \includegraphics[width=0.3\textwidth]{../images/data_samples/nontooth.png} & Prox. phalanx \\
        \hline
        \includegraphics[width=0.3\textwidth]{../images/data_samples/smudged.png} & P$\hat{\underline{\text{3}}}$ frag. (Crown) \\
        \hline
        \includegraphics[width=0.3\textwidth]{../images/data_samples/underlinedx.png} & ? M$\hat{\underline{\text{x}}}$ frag. \\
        \hline
    \end{tabular}
    \caption{Samples of input images and their corresponding labels.}
    \label{table:input_images}
\end{table}

\subsubsection{Data preprocessing}

\section{Figures}
Figure~\ref{fig:logo} gives an example how to add figures to the document. Remember always to cite the figure in the main text. There are many ways to cite, for example: University of Helsinki has a nice logo (see Fig.~\ref{fig:logo}).
\begin{figure}[h!] 
\centering 
\includegraphics[width=0.3\textwidth]{HY-logo-ml.png}
\caption{University of Helsinki flame-logo for Faculty of Science.\label{fig:logo}}
\end{figure}

\section{Tables}

Table~\ref{table:results} gives an example how to report experimental results. Remember always to cite the table in the main text. There are many ways to cite, for example: The results are as expected (see Table~\ref{table:results}).

\begin{table}
\centering
\caption{Experimental results.\label{table:results}}
\begin{tabular}{l||l c r} 
Koe & 1 & 2 & 3 \\ 
\hline \hline 
$A$ & 2.5 & 4.7 & -11 \\
$B$ & 8.0 & -3.7 & 12.6 \\
$A+B$ & 10.5 & 1.0 & 1.6 \\
\hline
%
\end{tabular}
\end{table}

\chapter{Citations}

\section{Citations to literature}

References are listed in a separate .bib-file. In this case it is named \texttt{bibliography.bib} with the following content:
\begin{verbatim}
@article{einstein,
    author =       "Albert Einstein",
    title =        "{Zur Elektrodynamik bewegter K{\"o}rper}. ({German})
        [{On} the electrodynamics of moving bodies]",
    journal =      "Annalen der Physik",
    volume =       "322",
    number =       "10",
    pages =        "891--921",
    year =         "1905",
    DOI =          "http://dx.doi.org/10.1002/andp.19053221004"
}
 
@book{latexcompanion,
    author    = "Michel Goossens and Frank Mittelbach and Alexander Samarin",
    title     = "The \LaTeX\ Companion",
    year      = "1993",
    publisher = "Addison-Wesley",
    address   = "Reading, Massachusetts"
}
 
@misc{knuthwebsite,
    author    = "Donald Knuth",
    title     = "Knuth: Computers and Typesetting",
    url       = "http://www-cs-faculty.stanford.edu/%7Eknuth/abcde.html"
}
\end{verbatim}

In the last reference url field the code \verb+%7E+ will translate into \verb+~+ once clicked in the final pdf.

References are created using command \texttt{\textbackslash cite\{einstein\}}, showing as \cite{einstein}. Other examples: \cite{latexcompanion,knuthwebsite}.

Citations should be arranged in alphabetical order by author, using the default style \texttt{abbrv}.



\section{Crossreferences}

Appendix~\ref{appendix:code} on page~\pageref{appendix:code} contains a code example.

\chapter{From tex to pdf}

In Linux, run \texttt{pdflatex filename.tex} and \texttt{bibtex
  filename} repeatedly until no more warnings are shown. You should
use \texttt{pdflatex} when compiling your document.
 
\chapter{Conclusions\label{chapter:conclusions}}

It is good to conclude with some insightful discussion. 

% STEP 5:
% Uncomment the following lines and set your .bib file and desired bibliography style
% to make a bibliography with BibTeX.
% Alternatively you can use the thebibliography environment if you want to add all
% references by hand.

\cleardoublepage %fixes the position of bibliography in bookmarks
\phantomsection

\addcontentsline{toc}{chapter}{\bibname} % This lines adds the bibliography to the ToC
\bibliographystyle{abbrv} % numbering alphabetic order
\bibliography{bibliography}

\begin{appendices}
\myappendixtitle

\chapter{Code example\label{appendix:code}}
Program code can be added as appendix:
\begin{verbatim}
#!/bin/bash          
text="Hello World!"
echo $text
\end{verbatim}

\end{appendices}

\end{document}
