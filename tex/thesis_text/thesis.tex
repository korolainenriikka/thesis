\documentclass{article}
\usepackage{graphicx}
\usepackage{amsmath}
\usepackage{amsthm}
\usepackage{amssymb}
\usepackage{geometry}
\usepackage[style=numeric,bibstyle=numeric,backend=biber,natbib=true,maxbibnames=99,giveninits=true,uniquename=init]{biblatex}
\usepackage[utf8]{inputenc}

\addbibresource{../bibliography.bib}
\title{Thesis notes}

\begin{document}

\tableofcontents

\section{Abstract}
Digitizing and uniformizing the structure of handwritten fossil records exhibits a great 
potential for increasing the accuracy of paleontological data analysis by increasing sample sizes. 
Approximately 90, 000 of such samples reside in the archives of the National Museum of Kenya, and 
an ongoing effort is to store this data in a digital format for better accessibility.
A previous project utilized a commercial optical character recognition service for automated reading of these catalogues. This generalist
handwriting detection model lacked the ability to detect special characters used to denote tooth samples, and could not utilize prior knowledge 
of the vocabulary that is more likely to be present in the data, leading to loss of information and detection mistakes.

This thesis aims to build a specialist character recognition model to increase the accuracy of 
the bone or tooth type specifying column of the digitized data by fine-tuning a state-of-the-art optical 
character recognition model with few-shot transfer learning. This is performed by first finding most accurate
recognition models, variants of convolutional neural networks or vision transformers, and most successful 
transfer learning methods for adapting a model to a new character set. Then, the character 
recognition accuracy of combinations of these methods are benchmarked using handlabeled image segments from the 
fossil catalogues. The final aim of this work is to use the best-performing model 
to obtain an accurate reading of the catalogues of the National Museum of Kenya, and publish the final model to be used 
by the paleontological community for further digitization efforts.


Keywords: Optical character recognition, Few-shot transfer learning, Vision transformers, Paleontological databases

\section{Introduction}

\section{Background}

\subsection{Neural Networks and Deep Learning}

\subsection{Fundamentals on paleoecology}

\subsubsection{Basics on ecology}

\subsubsection{Paleoenvironmental reconstruction}

\subsubsection{Diets and evolution}

\subsubsection{Composition of mammal teeth}

\section{data methods etc}

\subsection{data description}

\subsubsection{Notes on creating the dataset}

\subsubsection{Unicode characters used in data labeling}

% section todo: add images referred (one for images of chard + one for images & labels)

%tricks to label to make model work easier
To label the text found in cropped-out tooth fragment handwriting images, a few nonobvious 
conventions had to be set in place to construct a labeling system that can be assumed to 
be easier to learn for a machine learning model. The main guiding rule in these decisions was 
to encode each feature in the text in one consistent manner. What is meant by features and manners 
of denoting is explained next.

%\cite{unicode_homepage}
%expain unicode: graphene & code point 
%    explain: unicode has graphenes with code points. eg a is one graphene one code point,
%    à is one graphene two code points (dot on top and the letter). the top thing -like characters will be called 
%    "modifiers".
%general aim: one concept, one code point -> model able to figure out the connection between image and concept 
%concept: "number 2" "a character with a line on top"
%Also on the other hand using one modifier for all lowercase characters allows 
%the model to understand that there is a similarity between all lowercase characters.
%The intention is that one idea about a character is encoded as one code point, so that 
%the model can learn the mapping from the image of the character to the code point 
%combination
The unicode system \cite{unicode_homepage} constructs all known characters as signs called graphenes.
Each graphene can consist of any number of code points, with each code point having an unique identifier, denoted with "U+code point id".
Examples of graphenes with one code point are latin letters, such as 'K', special characters, such as '@', '\%' and '+',
or letters from different writing systems, such as %'ま', '朝' or 'Д'.
 Examples of multi-code point graphenes 
are latin letters with accents, such as '$\hat{\text{e}}$', or emoji characters with non-default skin tone, such as %'👍🏿'.
Code points added to the main code point, such as the circumflex accent '\^ ' are called modifiers.

The guiding principle in labeling the data was to encode each concept in the text as one unicode code point. A concept could be, for 
instance, the number two, or a character being positioned in subscript. The aim of this decision is to allow the model 
to find common image traits between characters of a similar type: a subscript character has dark pixels in lower positions, and shapes of all 
number two's have similar curvatures, for instance. As a second principle, it was chosen that each single character in the image, such as "letter C" 
or "a subscript four with a horizontal top line", would always be labeled as one graphene. 
These rules makes the encoding choices nonobvious: for example, 
a subscript number two would intuitively be labeled as the unicode code point '$_2$', but this was not done, 
since this graphene does not contain the code point for number two, 
and as a one code point graphene has no code point to extract to be used among the other subscript numbers.
Another intuitive choice, '\_2', would violate the one graphene per character rule.

%data characters:
%markings contain letters and numbers with no line, line on top or line at the bottom.
%Each character can be lower- or upper script.
%insert here example img of characters
The special characters in the dental fossil handwriting consist of sub- and superscript numbers, and characters with a horizontal line on 
top or bottom. Additionally, these two modifiers sometimes co-occur. Both denote which jaw the fragment is from: 
subscript and horizontal line on top of the character denote lower jaw, whereas superscript or line at the bottom of character 
signal upper jaw. In a few rare occurences, fractions are present 
to denote which proportion of the tooth is remaining in the sample. A sample of each type of notation can be found in Figure .
Note that ambiguous notations of for instance subscript number with a horizontal line at the bottom are allowed with this writing system.
The labeling notation chosen preserves the option to label these ambiguities.

%The modifiers used for these: 
%macron with lower ($\bar{\mathrm{A}}$) and upper variant.
%super/subscript: lower and upper script character set is incomplete for this purpose (eg 3 with upper macron and lower script needed)
%- from the model perspective 3 and $_3$ are no more similar than A and B, however, 
% therefore:  Therefore, modifier was used.
% there is no lower or upper case modifiers in unicode
%the caron ($\check{\mathrm{A}}$) was chosen as the lower script modifier, and the circumflex accent ($\hat{\mathrm{A}}$)
%as upper script. These were chosen since the arrow-like modifier pointing up or down
%is maybe the most logical placeholder for the missing modifier. More traditional 
%workarounds of missing upper or lower script, the underscore "\_" and separate 
%caret character "\^ " were not used because they would violate the one graphene one character rule

The following code points were chosen to denote the tooth marking system in the data labels.
The base code point modified with unicode modifiers was always chosen to be the latin letter or number present in the character. In the case of 
fractions, the number in the denominator was chosen as the base code point. The horizontal line on top of a character was denoted with the
combining macron modifier (U+0304, eg. $\bar{\text{A}}$), the line at the bottom respectively with the combining macron below (U+0331, eg. \underline{A}).
A subscript character was denoted with the combining caron (U+030C, eg. $\check{\mathrm{A}}$), and respectively the superscript with the combining
circumflex accent (U+0302, eg. $\hat{\text{A}}$). For fraction nominators, a modifier was chosen for each digit present in the dataset (TODO: add here after chosen).
These choices were made to improve human readability of the dataset, as the modifier choices are not relevant from the model perspective. A summary of the 
characters found and their labels can be found in image ....

\subsubsection{Data preprocessing}

\subsection{Methods}

\subsubsection{Encoding prior knowledge}

\section{results}

\section{conclusion}

\printbibliography

\end{document}