\documentclass{article}
\usepackage{graphicx}
\usepackage{amsmath}
\usepackage{amsthm}
\usepackage{amssymb}
\usepackage{geometry}
\usepackage[style=numeric,bibstyle=numeric,backend=biber,natbib=true,maxbibnames=99,giveninits=true,uniquename=init]{biblatex}

\addbibresource{../bibliography.bib}
\title{Master's thesis topic description: Fine-tuned optical character recognition for dental fossil markings}
\author{Riikka Korolainen}
\date{014926659}

\begin{document}


\maketitle

\section{General problem area}

paleoecology: data analysis on fossil data points

what we are able to learn: makeup of species of past ecosystems, reactions
of species to environmental changes

since 80's KNM has stored handwritten notes on found fossil specimens in 
Kenya/Ethiopia. approx 4,500 pages with approx 50 specimens in the catalogue

digitisation of hand-written fossil catalogues of the National Museum of Kenya

digitisation with Azure AI Vision services done, but that model could 
not read the special characters in the "element" column


\section{Research questions}

how well few-shot transfer learning methods perform at transfer from reading 
regular handwritten characters to reading charaters that have lower and 
upper script numbers

insert here image (ota element-sarake ja element csv ja toothrecords)

%\newgeometry{left=0.2cm,right=0.2cm}
%\begin{center}
%    \includegraphics*[scale=0.43]{data_sample.png}
%\end{center}
%\restoregeometry

\section{Methologies}

Literature review: find and compare the most successful OCR models for 
handwriting with bounding boxes given (we have them with azure vision)

Literature review: best few-shot transfer learning methods 

Hand-label data or request from Kenya 

Experiment: combinations of best OCR models + transfer techniques
keep track of experiments with MLflow

Train + store best method as a publicly available ML model. Submit to be 
used by KNM + stakeholders. Also run model on catalogues, get out cleaned
tooth records column

\section{Key references}

\begin{itemize}
    \item \cite{li2021trocr} presents a promising base model for fine-tuning
    \item \cite{Žliobaitė2023} the NOW database of fossil mammals
\end{itemize}

\printbibliography
\end{document}