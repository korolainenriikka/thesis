\documentclass{article}
\usepackage{graphicx}
\usepackage{amsmath}
\usepackage{amsthm}
\usepackage{amssymb}
\usepackage{geometry}
\usepackage[style=numeric,bibstyle=numeric,backend=biber,natbib=true,maxbibnames=99,giveninits=true,uniquename=init]{biblatex}

\addbibresource{../bibliography.bib}
\title{Master's thesis topic description: Fine-tuned optical character recognition for dental fossil markings}
\author{Riikka Korolainen}
\date{014926659}

\begin{document}


\maketitle

\section{General problem area}

paleoecology: data analysis on fossil data points

what we are able to learn: makeup of species of past ecosystems, reactions
of species to environmental changes, early human lifestyles \cite{Žliobaitė2023}

since 80's KNM has stored handwritten notes on found fossil specimens in 
Kenya/Ethiopia. approx 4,500 pages with approx 50 specimens in the catalogue. 

digitisation of hand-written fossil catalogues of the National Museum of Kenya
digitizing these will make all fossil data analysis more accurate
also allows integrating this data to larger collections of fossil data such as the NOW database \cite{Žliobaitė2023}

digitisation with Azure AI Vision services done, but that model could 
not read the special characters in the "element" column

insert here a sample image of element column content and then
regular expression solution

\section{Research questions}

how well few-shot transfer learning methods perform at transfer from reading 
regular handwritten characters to reading charaters that have lower and 
upper script numbers

%\newgeometry{left=0.2cm,right=0.2cm}
%\begin{center}
%    \includegraphics*[scale=0.43]{data_sample.png}
%\end{center}
%\restoregeometry

\section{Methodologies}

%Hand-label data or request from Kenya 

%Experiment: combinations of best OCR models + transfer techniques + 
%related problem solutions
%keep track of experiments with MLflow

%Train + store best method as a publicly available ML model. Submit to be 
%used by KNM + stakeholders. Also run model on catalogues, get out cleaned
%tooth records column

The thesis will consist of a literature review and an experimental section.
The literature review will consist of a synthesis of the relevant background 
information on deep learning, optical character recognition, transfer learning and 
paleoecology.
The main part of the literature review will consist of comparing
solutions to related problems of digitizing handwritten text that contains 
more unconventional characters. This part of the review can be divided into three partially overlapping review 
questions:

\begin{itemize}
    \item What is the best OCR model architecture?
    \item What is the best few-shot transfer learning method?
    \item Which solutions have previous works on related problems applied?
\end{itemize}

The goal of the literature review will be to choose a small set of solutions, which will be 
benchmarked in the experimental section. This part of the work will consist of attempting 
different combinations of approaches, and then comparing performance metrics. This will require a 
diligent experiment tracking system and hand-annotating data. The experiments will be performed 
using standard python data science libraries (pytorch, MLflow) and
 data from the fossil catalogues and specimen cards from the National Museum of Kenya. As a final deliverable,
 the fine-tuned model will be stored and made publicly available to be used by the museum.

\section{Key references}

\begin{itemize}
    \item \cite{li2021trocr} a promising base model for fine-tuning
    \item \cite{Žliobaitė2023} the NOW database of fossil mammals
    \item \cite{9151144} A survey on optical character recognition methods 
    \item \cite{10.1145/3582688} A survey on few-shot transfer learning
    \item \cite{Faith_Lyman_2019} A thorough reference and bibliography on paleoecology
\end{itemize}

\printbibliography
\end{document}